\documentclass[xcolor=pdftex,x11names,table]{beamer}
\usepackage[T1]{fontenc}
\usepackage[utf8]{inputenc}
\usepackage[frenchb]{babel}
\usepackage{colortbl}

\usepackage{tikz}
\usetikzlibrary{decorations,arrows,shapes,backgrounds,shadows,calc}
\usetikzlibrary{positioning}
\usetikzlibrary{automata}

\usepackage{ifthen}
\usepackage{changepage}

\usepackage{listings}

\setbeamertemplate{blocks}[rounded][shadow=true]

\title[Persistance des données]{Persistance des données en Java}
%\author{Sébastien NEDJAR}
%\institute{IUT d'Aix-en-Provence}
\date{}

\usecolortheme{crane}
\definecolor{couleurprincipale}{HTML}{FFCC00}

\AtBeginSection[]
{
  \begin{frame}<beamer>
    \frametitle{Plan}
    \tableofcontents[currentsection,hideothersubsections]
  \end{frame}
}

\begin{document}
  \lstset{             
    breaklines=true,                                     % line wrapping on
    %frame=ltrb,
    %samepage=true,
    language=Java,
    basicstyle=\normalsize,
    frameround=ftft,
    keywordstyle=\ttfamily\color{SeaGreen4},
    identifierstyle=\ttfamily\bfseries\color{RoyalBlue4},
    commentstyle=\color{RoyalBlue3},
    stringstyle=\ttfamily,
    showstringspaces=false,
    tabsize=2,  
  }

  % For every picture that defines or uses external nodes, you'll have to
  % apply the 'remember picture' style. To avoid some typing, we'll apply
  % the style to all pictures.
  \tikzstyle{every picture}+=[remember picture]
  \tikzstyle{na} = [baseline=-.5ex]

  % ---------- Titre -----------
  \begin{frame}
  \titlepage
  \end{frame}
  % ----------------------------
  \section{Présentation du cours}
    \begin{frame}{Introduction}
      En programmation, la gestion de persistance des données se réfère aux mécanismes 
      responsables de la sauvegarde et la restauration de données, afin que l'état ou les données 
      d'un programme puissent lui survivre dans le temps et dans l'espace.
      \pause 
      \begin{itemize}[<+->]
      	\item La première solution simple de persistance est l'utilisation des fichiers (binaires, textes, plats, XML, ...)
      	\item Pour une application répartie, les fichiers ne suffisent plus. 
      	  Il faut donc une solution pour centraliser toutes les données persistantes.
      	\item Dans la majorité des cas on souhaite déléguer la gestion des données 
      	  persistantes à une base de données relationnelle.
      \end{itemize}
    \end{frame}
    
    \begin{frame}{Pourquoi une BD relationnelle ?}
      \begin{itemize}
        \item Une théorie solide et des normes reconnues.
        \item Facilité d'accès et généricité du langage de requête SQL.
        \item Grande efficacité pour effectuer des recherches complexes 
        dans des grandes bases de données.
        \item Sécurité et intégrité des données.
        \item Transaction et gestion de la concurrence.
        \item Faculté de gestion d'énormes volumes de données. 
        \item Technologie largement dominante dans toute l'industrie informatique pour la gestion des données.
        \item Large offre commerciale répondant à un très large panel de besoins.
      \end{itemize}
      \pause
      La question que l'on va se poser dans ce cours sera donc "comment gérer \emph{proprement} la persistance des données 
      avec un SGBD-R dans une application développée dans un langage orienté objet comme Java ?"
    \end{frame}
    
    \begin{frame}{Limites des BD relationnelles}
      \begin{itemize}
        \item Les données des applications sont modélisées en objet et non en relationnel.
        \item Le relationnel est moins riche que le modèle objet : Pas d'héritage, 
        pas de composition ni d'attribut multivalué.
        \item La modélisation est faite pour des données ayant une structures quasi immuable. Il est très coûteux de 
        modifier le schéma d'une table en production. Le modèle est peu adapté aux données faiblement structurées.
        \item La situation oligopolistique du marché des gros SGBD-R implique un coût d'accès prohibitif à certaines technologies 
        (BD réparties, Caches, Clusters, ...).
      \end{itemize}
      Toutes ces limites ont permis l'émergence récente d'un nouveau type de base de données : les BD "\emph{NoSQL}" 
      (Not Only SQL). Elles sont très en vogue ces derniers temps car elles comblent un besoin marginal mais non 
      satisfait par les BD relationnelles.
    \end{frame}
    
    \begin{frame}{Applications réparties : l'architecture 3-tiers}
      L'architecture 3-tiers, également connue sous le nom de client-serveur distribué, sépare l'application en 
      trois niveaux de services distincts :
      \begin{itemize} 
        \item premier niveau : l'affichage et les traitements locaux (contrôles de saisie, mise en forme des données
        etc.) sont pris en charge par le client;
        \item deuxième niveau : les traitements applicatifs globaux relatifs au domaine de l'application (traitements « métiers »).
        Ils sont pris en charge par le service applicatif;
        \item troisième niveau : les services de base de données sont pris en charge par un SGBD.
      \end{itemize}
      Généralement les différents niveaux sont nommés respectivement ainsi : la couche présentation, la couche métier et la couche de persistance.
    \end{frame}
    
    \begin{frame}{Persistance en Java}
      La plate-forme Java met à disposition des développeurs plusieurs technologies pour accéder aux données d'un SGBD-R :
      \pause
      \begin{itemize} 
        \item Outils fondés sur les concepts relationnels (Relation, Tuple, ...)
          \begin{itemize} 
            \item JDBC
          \end{itemize}\pause
          \item Outils fondés sur les concepts objets (Classe, Objet, ...)
          \begin{itemize} 
            \item JDO
            \item Toplink
            \item MyBatis
            \item JPA 2 (Java Persistence API)
              \begin{itemize}
                \item EclipseLink
                \item Hibernate
                \item OpenJPA
              \end{itemize}
          \end{itemize}                            
      \end{itemize}
      \pause
      Dans ce cours nous allons étudier un produit de chaque famille pour comprendre concrètement leurs avantages et 
      leurs limites respectives.        
    \end{frame}

  \section{Persistance fondée sur les concepts relationnels : JDBC}
  \subsection{Introduction}
    \begin{frame}{Java DataBase Connectivity}
      JDBC (Java DataBase Connectivity) est une interface de programmation créée par Sun Microsystems, 
      pour les programmes utilisant la plate-forme Java. Il permet aux applications Java d'accéder par le biais d'une 
      interface commune à des sources de données tabulaires (principalement relationnelles) pour lesquelles il existe 
      un pilote JDBC.\pause
      \vfill
      Les objectifs de JDBC sont les suivants : 
      \begin{itemize}
        \item Fournir une interface uniforme permettant un accès homogène aux SGBD
        \item Être simple à mettre en œuvre.
        \item Être indépendant du SGBD cible en n'utilisant que les fonctionnalités fournies par SQL.
      \end{itemize}
    \end{frame}
  
    \begin{frame}{Architecture de JDBC}
      \begin{columns}[c]
        \begin{column}{0.5\paperwidth}
          JDBC est constitué de 3 couches logicielles distinctes :  
          \begin{itemize}
          	\item<2-> Une API constituée principalement d'interfaces (au sens Java) qui 
          	sera utilisée par le code client.
            \item<3-> Le Driver Manager qui s'occupe de dialoguer avec le pilote propre au SGBD choisi.
            \item<4-> Le Driver (ou pilote) qui a pour rôle de transformer les appels en ordres compréhensibles par le SGBD cible.
          \end{itemize}
        \end{column}
        \begin{column}{0.4\paperwidth}
          \begin{tikzpicture}[node distance=0.7cm,trans/.style={thick,<->,shorten >=2pt,shorten <=2pt,>=stealth}]
              \setbeamercovered{invisible}
               
               \node[draw, rectangle,minimum width=80] at(0,0) (code) {Application};\pause
               \node[draw, rectangle,minimum width=80] (jdbc_api) [below = of code] {JDBC API};
               \draw[trans] (code) -- (jdbc_api);\pause
 
               \node[draw, rectangle,minimum width=80] (driver_manager) [below =0mm of jdbc_api] {Driver Manager};\pause              
               \node[draw, rectangle,minimum width=80] (driver) [below =0mm of driver_manager] {Driver};\pause
               \node[draw, cylinder, alias=cyl, shape border rotate=90, aspect=1.6, %
                     minimum height=45, minimum width=80, outer sep=-0.5\pgflinewidth, %
                      color=orange!40!black, left color=orange!70, right color=orange!80,%
                      middle color=white] (sgbd) [below=1cm of driver] {};
                      
               \node at (sgbd) {SGBD};
               
               \draw[trans] (sgbd) -- node[pos=0.5] (milieu) {} (driver);
               \path (driver.east|-milieu) -- ++(5mm,0) node (bord_d) {};
               \path (driver.west|-milieu) -- ++(-5mm,0) node (bord_g) {};
               \draw[thick,dashed] (bord_g) -- (bord_d);
               
          \end{tikzpicture}
        \end{column}       
      \end{columns}    
    \end{frame}

    \begin{frame}{API JDBC}
      L'API JDBC permet principalement de faire 3 types de choses : 
      \begin{itemize}
        \item Établir une connexion avec une BD ou toute autre source de données tabulaires.
        \item Envoyer des ordres écrits en SQL.
        \item Traiter les résultats.
      \end{itemize}\pause
      Elle est fournie par les packages \lstinline$java.sql$ et \lstinline$javax.sql$. 
      Ils contiennent les principales interfaces suivantes : 
      {\tt 
      \begin{itemize}
        \item \lstinline$Statement$, \lstinline$CallableStatement$, \lstinline$PreparedStatement$
        \item \lstinline$DatabaseMetaData$, \lstinline$ResultSetMetaData$
        \item \lstinline$ResultSet$
        \item \lstinline$Connection$, \lstinline$Driver$
      \end{itemize}
      }
    \end{frame}

    \begin{frame}{API JDBC : interfaces principales}
      \begin{itemize}
        \item \lstinline$Driver$: renvoie une instance de \lstinline$Connection$
        \item \lstinline$Connection$: connexion à une base
        \item \lstinline$Statement$: instruction SQL
        \item \lstinline$PreparedStatement$: instruction SQL paramétrée
        \item \lstinline$CallableStatement$: procédure stockée dans la base
        \item \lstinline$ResultSet$: ensemble de tuples retourné par une requête SQL
        \item \lstinline$ResultSetMetaData$: description des tuples récupérés
        \item \lstinline$DatabaseMetaData$: informations sur la base de données
      \end{itemize}    
    \end{frame}    
    
    \begin{frame}{Traitement d’un ordre SQL}
      Le scénario pour traiter un ordre SQL avec JDBC est le suivant :
      \begin{enumerate}
        \item Importer l'API.
        \item Enregistrer le pilote JDBC (Optionnel depuis JDBC 4.0).
        \item Connexion à la base de données.
        \item Création d’une instruction SQL.
        \item Exécution de la requête.
        \item Traitement de l’ensemble des résultats.
        \item Libération des ressources et fermeture de la connexion.
      \end{enumerate}
      Étant donné que chaque étape est susceptible de rencontrer des erreurs, 
      il faut rajouter une étape de gestion des exceptions.
    \end{frame}
		\lstset{basicstyle=\scriptsize,language=Java}
		
    \subsection{Traitement d’une requête SQL}
    
    \begin{frame}[fragile]
			\frametitle{Traitement d’une requête SQL}
			\begin{lstlisting}
import java.sql.*;// Import de l'API
public class App {
  static String req="SELECT ID, NAME FROM ARTIST";
  public static void main(String[] args){
    try {
      //Connexion a la BD
      Connection conn = DriverManager.getConnection("jdbc:oracle:...","user","pwd");
      //Creation d'une instruction
      Statement stmt = conn.createStatement();
      //Execution de la requete
      ResultSet rset = stmt.executeQuery(req);
      //Traitement des resultats
      while (rset.next()){
        System.out.println(rset.getString("NAME"));
      }
      stmt.close();//Liberation des ressources
      conn.close();//Deconnexion de la BD
    } catch (SQLException e) {
      e.printStackTrace();//Arggg!!!
    }
  }
}
			\end{lstlisting}
		\end{frame}
		
		\lstset{basicstyle=\normalsize}
		\begin{frame}{Interface \lstinline$Driver$}
			\begin{itemize}
	      \item La méthode \lstinline$connect()$ de \lstinline$Driver$
	   		  \begin{itemize}
	      		\item Prend en paramètre une URL vers la BD.
	      		\item Retourne un objet \lstinline$Connection$ qui permettra de créer des requêtes.
	      	\end{itemize}
	      \item URL pour accéder à la base (syntaxe dépend du SGBD cible)
	      	\begin{itemize}
	      		\item \texttt{jdbc:<sous-protocole>:<nom-BD>?param=valeur, ...}
	      		\item sous-protocole: \texttt{oracle:thin}
	      		\item nom-BD: \texttt{@//allegro.iut.univ-aix.fr:1522/orcl}
	      		\item Par exemple :
					  	\begin{itemize}
					  		\item \texttt{jdbc:oracle:thin:@//allegro:1522/orcl} 
					  		\item \texttt{jdbc:derby://localhost:1527/beeDB}
					  		\item \texttt{jdbc:mysql://localhost:3306/FloyDB}
					  	\end{itemize}
	      	\end{itemize}
			\end{itemize}
		\end{frame}

    \begin{frame}
    \frametitle{Étape 2 : enregistrer le pilote}
    	Cette étape est automatique depuis la version 4.0 de JDBC. Si vous ne disposez pas d'un driver 
    	compatible l'enregistrement est manuel.
    	\begin{itemize}
	      \item La classe \lstinline$DriverManager$ a pour but de gérer les différents drivers (instances de Driver).
	      \item Pour qu'un driver soit disponible, il faut charger sa classe en mémoire. Pour que la JVM accepte 
	      de le charger il faut que son archive \emph{jar} soit dans le \texttt{CLASSPATH}.
    	\end{itemize}
    	\pause
    	\vfill
    	\begin{block}{}
    	\lstinline$Class.forName("oracle.jdbc.driver.OracleDriver");$
    	\end{block}
    	\vfill
    	\pause
    	\begin{itemize}
	      \item Cette instruction crée une instance du driver et l'enregistre auprès de la classe \lstinline$DriverManager$.
    	\end{itemize}
    \end{frame}
    
    \begin{frame}
    \frametitle{Étape 3 : connexion à la base de données}
    	Pour se connecter, on utilise la méthode \lstinline$getConnection()$ de \lstinline$DriverManager$. 
    	Elle admet trois \lstinline$String$ paramètres :
    	\begin{itemize}
    		\item L’URL de la base de données.
				\item Le nom d'utilisateur.
				\item Le mot de passe de l'utilisateur.
			\end{itemize}
   		\pause
			\begin{block}{}
   			   	\lstinline$Connection conn = DriverManager.getConnection(url, user, pwd);$
 			\end{block}
    	\pause
    	\begin{itemize}
	      \item Le \lstinline$DriverManager$ essaye tous les drivers enregistrés (chargés en mémoire avec \lstinline$Class.forName()$) 
	      jusqu’à ce qu’il trouve un driver qui lui fournisse une instance de \lstinline$Connection$.
	      \item Les connexions sont des ressources coûteuses et longues à obtenir, il faut donc éviter 
	      d'en ouvrir une pour chaque requête à exécuter.
    	\end{itemize}  
    \end{frame}

    \begin{frame}
    \frametitle{Étape 4 : création d’une instruction SQL}
		  L'interface \lstinline$Statement$ possède les méthodes nécessaires pour réaliser les requêtes sur 
		  la base associée à la connexion dont il dépend. Il existe 3 types de \lstinline$Statement$ :\pause
		  \begin{enumerate}[<+->]
				\item Les \lstinline$Statement$ : Ils permettent d'exécuter n'importe quelle requête sans paramètre. La requête est 
				interprétée par le \textsc{Sgbd} au moment de son exécution. Ce type d'ordre est à utiliser principalement 
				pour les requêtes à usage unique.
				\item Les \lstinline$PreparedStatement$ : Ils permettent de précompiler un ordre avant son exécution. Ils sont 
				particulièrement importants pour les ordres destinés à être exécutés plusieurs fois comme par exemple les 
				requêtes paramétrées.
				\item Les \lstinline$CallableStatement$ : Ils sont destinés à l'appel des procédures stockées.
			\end{enumerate}
    \end{frame}

    \begin{frame}
    \frametitle{Étape 4 : création d’une instruction SQL}
	    À partir d'une instance de l’objet \lstinline$Connection$, on récupère une nouvelle instance de \lstinline$Statement$.
	    \begin{block}{}
    		\lstinline$Statement s1=connexion.createStatement();$\\
        \lstinline$PreparedStatement s2=connexion.prepareStatement(req);$\\
        \lstinline$CallableStatement s3=connexion.prepareCall(req);$
      \end{block}
    \end{frame}

    \begin{frame}
    \frametitle{Étape 5 : Exécution de la requête}
    	Afin d’exécuter une requête, il suffit de faire appel à l’une des méthodes \lstinline$executeXXXX()$
    	de l’objet \lstinline$Statement$ que l’on vient de créer. Il existe 3 types d'exécution : 
    	\begin{itemize}
    		\item Les interrogation de données : on utilise la méthode \lstinline$executeQuery()$ en lui passant 
				en paramètre une chaîne de caractères (\lstinline$String$) contenant la requête. Cette méthode retourne 
				un objet du type \lstinline$ResultSet$ contenant l'ensemble des résultats de la requête.
        \item Les modifications de données (INSERT, UPDATE, DELETE) :  exécutés avec la
				méthode \lstinline$executeUpdate()$ qui retourne un entier correspondant au nombre de lignes impactées par la mise à jour.
        \item Nature inconnue, plusieurs résultats ou procédures stockées : on utilise la méthode \lstinline$execute()$.
      \end{itemize}
    \end{frame}
    
    \begin{frame}
    \frametitle{Étape 6 : Traitement de l’ensemble des résultats}
    	\begin{itemize}
    		\item \lstinline$executeQuery()$ renvoie une instance de	\lstinline$ResultSet$
    		\item \lstinline$ResultSet$ va permettre de parcourir toutes les lignes renvoyées par le SELECT
    		\item Au début, \lstinline$ResultSet$ est positionné avant la première ligne et il faut donc commencer par le faire avancer 
    		à la première ligne en appelant la méthode \lstinline$next()$
    		\item Cette méthode permet de passer à la ligne suivante ; elle renvoie \lstinline$true$ si cette ligne suivante existe et \lstinline$false$ sinon.
			\end{itemize}
			\begin{block}{}
				\lstinline$ResultSet rset = stmt.executeQuery(req);$\\
				\lstinline$while (rset.next())$\\
				\lstinline$\ \ \ \ System.out.println(rset.getString(2));$\\
      \end{block}
    \end{frame}
    
    \begin{frame}
    \frametitle{Différents types de \lstinline$ResultSet$}
      Il existe quatre types de \lstinline$ResultSet$:
    	\begin{description}
    		\item[Scroll-insensitive:] Vision figée du résultat de la requête au moment de son évaluation.
        \item[Scroll-sensitive:] Le \lstinline$ResultSet$ montre l'état courant des données (modifiées/détruites).
        \item[Read-only:] Pas de modification possible (version par défaut) donc un haut niveau de concurrence.
        \item[Updatable:] Possibilité de modification donc pose de verrou et faible niveau de concurrence.
			\end{description}
			\begin{block}{}
        \lstinline$Statement stmt = con.createStatement( $\\
        \lstinline$\ \ ResultSet.TYPE_SCROLL_INSENSITIVE, $\\
        \lstinline$\ \ ResultSet.CONCUR_UPDATABLE); $\\
        \lstinline$ResultSet rs=stmt.executeQuery("SELECT * FROM ARTIST");$\\
      \end{block}
      Ce \lstinline$ResultSet$ est modifiable mais il ne reflète pas les modifications faites par d'autres transactions.
    \end{frame}
     
    \begin{frame}
    \frametitle{Déplacement dans un \lstinline$ResultSet$}
      Depuis JDBC 3.0, on peut se déplacer à l'intérieur d'un \lstinline$ResultSet$. Ci dessous sont 
      listées les méthodes de déplacement :
      \begin{itemize}
    		\item \lstinline$rs.first();$
    		\item \lstinline$rs.beforeFirst();$
    		\item \lstinline$rs.next();$
    		\item \lstinline$rs.previous();$
    		\item \lstinline$rs.afterLast();$
    		\item \lstinline$rs.absolute( n );$
    		\item \lstinline$rs.relative( n );$
    	\end{itemize}
    	Les tuples de l'ensemble de résultats n'étant chargés qu'à la demande(avec un préchargement), 
    	un parcours aléatoire peut s'avérer très coûteux. 
    \end{frame}  
    
    \begin{frame}
    \frametitle{Modification d'un ResultSet}
      Modification :
			\begin{block}{}
        \lstinline$rs.absolute(100);$\\
        \lstinline$rs.updateString("NAME", "RICHARD");$\\
        \lstinline$rs.updateRow();$\\
      \end{block}
      Destruction: 
      \begin{block}{}
        \lstinline$rs.deleteRow();$\\
      \end{block}
      Insertion de lignes:
      \begin{block}{}
        \lstinline$rs.moveToInsertRow();$\\
        \lstinline$rs.updateString("NAME", "ROGER");$\\
        \lstinline$rs.insertRow();$\\
        \lstinline$rs.first();$\\
      \end{block}
    \end{frame}  
    
    \begin{frame}
    \frametitle{Étape 6 : Traitement de l’ensemble des résultats}
    	\begin{itemize}
    		\item Quand un \lstinline$ResultSet$ est positionné sur une ligne, les méthodes \lstinline$getXXX$ permettent de récupérer
				les valeurs des colonnes de la ligne :
				    	\begin{itemize}
    						\item \lstinline$getXXX(int numeroColonne)$ (position dans le SELECT de la valeur à récupérer)
    						\item \lstinline$getXXX(String nomColonne)$ (nom simple d’une colonne, pas préfixé par un nom de table; dans le cas d’une jointure utiliser un alias de colonne)
							\end{itemize}
				\item \lstinline$XXX$ désigne le type Java de la valeur que l'on va récupérer, par exemple \lstinline$String$, \lstinline$int$ ou \lstinline$double$.
				\item Par exemple, \lstinline$getInt$ renvoie un \lstinline$int$.
			\end{itemize}\pause
			\begin{block}{}
				\lstinline$ResultSet rset = stmt.executeQuery(req);$\\
				\lstinline$while (rset.next())$\\
				\lstinline$\ \ \ \ System.out.println(rset.getInt(1));$\\
      \end{block}
    \end{frame}
    \begin{frame}
    \frametitle{Correspondance de types entre Java et SQL}
    	\begin{itemize}
    		\item Tous les SGBD n’ont pas les mêmes types SQL; 
    		même les types de base peuvent présenter des différences importantes.
				\item Pour cacher ces différences, JDBC définit ses propres types SQL 
				dans la classe \lstinline$Types$.
				\item Le driver JDBC fait la traduction de ces types dans les types du SGBD.
				\item Il reste le problème de la correspondance entre les types Java et les types SQL.
				C'est le rôle de \lstinline$getXXX$ et de \lstinline$setXXX$. 
				\item Par exemple, \lstinline$getString$ indique que l’on veut récupérer la donnée SQL dans une \lstinline$String$.
				\item Si le \lstinline$Driver$ n'arrive pas à effectuer la conversion, il lève une exception.
    	\end{itemize}
    \end{frame}

    \begin{frame}
    \frametitle{Correspondance de types entre Java et SQL}
    	\begin{center}
	      \rowcolors{2}{couleurprincipale!20}{couleurprincipale!10} 
	      \begin{tabular}{c|c}
	        \hline
	        \rowcolor{couleurprincipale!80}
					SQL &       Java  \\ \hline
					CHAR&       \lstinline$String$\\
					VARCHAR& 	  \lstinline$String$\\
					NUMERIC& 	  \lstinline$java.math.BigDecimal$\\
					DECIMAL& 	  \lstinline$java.math.BigDecimal$\\
					BIT& 	      \lstinline$boolean$\\
					TINYINT& 	  \lstinline$byte$\\
					SMALLINT& 	\lstinline$short$\\
					INTEGER &	  \lstinline$int$\\
					BIGINT& 	  \lstinline$long$\\
					REAL& 	    \lstinline$float$\\
					FLOAT& 	    \lstinline$double$\\
					DOUBLE& 	  \lstinline$double$\\
					BINARY& 	  \lstinline$byte[]$\\
					VARBINARY& 	\lstinline$byte[]$\\
			        	\hline
			  \end{tabular}
	    \end{center}   
    \end{frame}

    \begin{frame}
    \frametitle{Correspondance de type entre Java et SQL}
    	\begin{center}
	      \rowcolors{2}{couleurprincipale!20}{couleurprincipale!10} 
	      \begin{tabular}{c|c|c}
	        \hline
	        \rowcolor{couleurprincipale!80}
					SQL &      Java & Explication  \\ \hline
					DATE&      \lstinline$java.sql.Date$& 	codage de la date\\
					TIME&      \lstinline$java.sql.Time$& 	codage de l'heure\\
					TIMESTAMP& \lstinline$java.sql.TimeStamp$& 	codage de la date et de l'heure\\ \hline
			  \end{tabular}
	    \end{center}   
    \end{frame}

    \begin{frame}
    \frametitle{Gestion des valeurs nulles}
			Comment reconnaître dans Java le cas d’une valeur NULL de SQL ? \\
			Pour les méthodes \lstinline$getXXX()$, la convention est la suivante :
    	\begin{itemize}
    		\item Les méthodes \lstinline$getString()$, \lstinline$getObject()$, ..., \lstinline$getDate()$ retournent une référence \lstinline$null$.
				\item Les méthodes \lstinline$getInt()$, \lstinline$getByte()$, ..., \lstinline$getShort()$ retournent la valeur $0$.
				\item La méthode \lstinline$getBoolean()$ renvoie la valeur \lstinline$false$.
  		\end{itemize}
  		Cette convention implique que l'on ne peut pas distinguer un \texttt{NULL} d'un zéro.
  		La méthode \lstinline$wasNull()$ de \lstinline$ResultSet$ permet de vérifier si le dernier \lstinline$getXXX()$ 
  		a retourné un \texttt{NULL}.
    \end{frame}
    
    \begin{frame}
    \frametitle{Étape 7 : Libération des ressources}
    %\transboxin
    	\begin{itemize}
    		\item Toutes les ressources JDBC doivent être fermées dès qu’elles ne sont plus utilisées.
    		\item Le plus souvent, la fermeture doit se faire dans un bloc \lstinline$finally$ pour qu’elle ait 
    		lieu quel que soit le déroulement des opérations (avec ou sans erreurs).
				\item Un \lstinline$Statement$ ou un \lstinline$ResultSet$ fermé ne peut plus être utilisé. La fermeture d'un 
				\lstinline$Statement$ ferme automatiquement les \lstinline$ResultSet$ qui lui sont associés.
				\item La \lstinline$Connection$ étant une ressource coûteuse, elle doit être impérativement fermée. 
				Si possible utilisez un \emph{pool} de connexion.
			\end{itemize}
			\begin{block}{}
				\lstinline$rset.close();$\\
				\lstinline$stmt.close();$\\
				\lstinline$conn.close();$
      \end{block}
    \end{frame}

    \begin{frame}
    \frametitle{Étape 8: Gestion des exceptions}
    	\begin{itemize}
    		\item Une \lstinline$SQLException$ (ou un objet héritant de cette classe) est levée dès qu’une connexion 
    		ou un ordre SQL ne se déroule pas correctement.
    		  \begin{itemize}
    				\item La méthode \lstinline$getMessage()$ permet d'obtenir le message d'erreur en clair.
					\end{itemize}
				\item Pour distinguer des types d’exception, JDBC a introduit 3 sous-classes de \lstinline$SQLException$ :
				  \begin{itemize}
    				\item \lstinline$SQLNonTransientException$ : le problème ne peut être résolu sans une action externe ; 
    				inutile de réessayer la même action sans rien faire de spécial.
    				\item \lstinline$SQLTransientException$ : le problème peut avoir été résolu si on attend un peu avant 
    				d’essayer à nouveau.
    				\item \lstinline$SQLRecoverableException$ : l’application peut résoudre le problème en exécutant une 
    				certaine action mais elle devra fermer la connexion actuelle et en ouvrir une nouvelle.
					\end{itemize}
			\end{itemize}
		\end{frame}
		
		\begin{frame}
    \frametitle{Étape 8: Gestion des exceptions}
    	\begin{itemize}
    		\item Une requête SQL peut provoquer plusieurs exceptions
    		\item On peut obtenir la prochaine exception par la méthode \lstinline$getNextException()$
    		\item Une exception peut avoir une cause ; on l'obtient par la méthode \lstinline$getCause()$
    		\item Toutes ces exceptions peuvent être parcourues par une boucle « for-each » :\\
    		\lstinline$for (Throwable e : ex) \{...\}$ 
			\end{itemize}
		\end{frame}
		
		\subsection{Traitement d’un ordre de modification des données}
		\begin{frame}[fragile]
    \frametitle{Insertion de tuples}
      Pour insérer un nouveau tuple dans la BD, il suffit d'exécuter l'instruction \texttt{INSERT} de la manière suivante :
      \pause
      \begin{block}{}
    	  \begin{lstlisting}
Statement st = conn.createStatement();
int nb = st.executeUpdate(
    "INSERT INTO ARTIST(ID, NAME) " +
    "VALUES (" + id + ", '" + nom + "')"
    );
System.out.println(nb + " tuple insere");
st.close();
			  \end{lstlisting}
			\end{block}
			Le principe est identique pour les instructions \texttt{UPDATE} et \texttt{DELETE}.\\
			Si l'on doit exécuter un grand nombre de modifications ce genre de solution est coûteuse 
			et difficile à mettre en œuvre.		
		\end{frame}

		\begin{frame}[fragile]
    \frametitle{Ordre SQL paramétrée}
      Les paramètres dans un ordre SQL permettent d'économiser des ressources en évitant de recompiler 
      plusieurs fois les mêmes requêtes.
      \pause
      \begin{block}{}
    	  \begin{lstlisting}
PreparedStatement st = conn.prepareStatement(
    "UPDATE ARTIST SET AGE = ? " +
    "WHERE NAME = ? ");
for( ... ) {
    st.setInt(1, age[i]);
    st.setString(2, nom[i]);
    st.execute();
}       
        \end{lstlisting}
			\end{block}
			On remarquera l'utilisation de la méthode \lstinline$execute()$
		\end{frame}
		
		
		\subsection{Traitement d’un appel de procédure stockée}
		\begin{frame}[fragile]
    \frametitle{Traitement d’un appel de procédure stockée}
      Lorsqu'un traitement peut être fait directement par le SGBD, on utilisera les procédures stockées.\pause  
      \begin{block}{}
    	  \begin{lstlisting}
CallableStatement s = conn.prepareCall(
    "{call ma_procedure[(?,?)]}");
// fixer le type de parametre de sortie
s.registerOutParameter(2,java.sql.Types.FLOAT);
//fixer la valeur du parametre
s.setInt(1,valeur);
s.execute();
System.out.println("res = " + s.getFloat(2));
        \end{lstlisting}
			\end{block}
			Ce type de traitement permet d'économiser des ressources en évitant beaucoup de transferts de données inutiles.
		\end{frame}
		
		\begin{frame}
    \frametitle{Gestion des transactions}
      Par défaut, les connections JDBC fonctionnent en mode "Auto Commit". Pour pouvoir utiliser les transactions, il 
      faut au préalable désactiver ce mode pour la connexion courante.
    	\begin{itemize}
    		\item \lstinline$conn.setAutoCommit(false);$
    		\item \lstinline$conn.commit();$
    		\item \lstinline$conn.rollback();$
			\end{itemize}
			Combinée avec une bonne gestion des erreurs, les transactions permettent de garantir l'intégrité des données.
		\end{frame}
		
		\subsection{Les Meta Informations}
		\begin{frame}
    \frametitle{Les Meta Informations}
    	\begin{itemize}
    		\item JDBC permet de récupérer des informations sur le type de données que l'on vient de récupérer par un 
    		\texttt{SELECT} (interface \lstinline$ResultSetMetaData$),
    		\item mais aussi sur la base de données elle-même (interface \lstinline$DatabaseMetaData$).
    		\item Les données que l’on peut récupérer avec \lstinline$DatabaseMetaData$ dépendent du SGBD avec lequel 
    		on travaille.
			\end{itemize}
		\end{frame}
		
		\begin{frame}
    \frametitle{Méta Informations sur les Result Set}
      \begin{block}{}
        \lstinline$ResultSetMetaData rsmd = rs.getMetaData();$
      \end{block}
    	Informations disponibles :
    	\begin{itemize}
    		\item nombre de colonnes (\lstinline$getColumnCount()$),
        \item libellé d'une colonne (\lstinline$getColumnName(int column)$),
        \item table d'origine (\lstinline$getTableName(int column)$),
        \item type associé à une colonne (\lstinline$getColumnType(int column)$),
        \item la colonne est-elle nullable ? (\lstinline$isNullable(int column)$)
        \item etc.
			\end{itemize}
			Permet d'écrire un code plus générique s'adaptant à un plus grand nombre de requêtes.
		\end{frame}
		
		\begin{frame}[fragile]
    \frametitle{Méta Informations sur les Result Set}
      \begin{block}{}
    	  \begin{lstlisting}
ResultSet rs = 
  stmt.executeQuery("SELECT * FROM ARTIST");
ResultSetMetaData rsmd = rs.getMetaData();
int nbCol = rsmd.getColumnCount();
//Affichage de la liste des colonnes
for (int i = 1; i <= nbCol; i++) {
  String typeCol = rsmd.getColumnTypeName(i);
  String nomCol = rsmd.getColumnName(i);
  System.out.println("Colonne " + i 
    + " de nom " + nomCol 
    + " de type " + typeCol);
}
        \end{lstlisting}
			\end{block}
		\end{frame}
		
		\begin{frame}
    \frametitle{Méta Informations sur la BD}
      \begin{block}{}
        \lstinline$DataBaseMetaData metaData = conn.getMetaData();$
      \end{block}
      Informations disponibles :
    	\begin{itemize}
    		\item tables existantes dans la base (\lstinline$getTables()$),
    		\item nom d'utilisateur (\lstinline$getUserName()$),
    		\item version du pilote (\lstinline$getJDBCMajorVersion()$ et \lstinline$getJDBCMinorVersion()$),
    		\item prise en charge des jointure externes ? (\lstinline$supportsOuterJoins()$),
    		\item etc.
			\end{itemize}
			Ces informations permettent d'adapter les traitements JDBC aux capacité du SGBD cible. Le code devient ainsi plus 
			portable.
		\end{frame}
		
		\begin{frame}[fragile]
    \frametitle{Méta Informations sur la BD}
      \begin{block}{}
    	  \begin{lstlisting}
metaData = conn.getMetaData();
String[] types = { "TABLE", "VIEW" };
String nomTables;
//Recuperation de la liste des tables
ResultSet rs =
  metaData.getTables(null, null, "%", types);
//Traitement des resultats
while (rs.next()) {
  nomTable = rs.getString(3);
  System.out.println(nomTable);
}
        \end{lstlisting}
			\end{block}
		\end{frame}
	
		
		\subsection{Conclusion Temporaire}
    \begin{frame}
    \frametitle{Conclusion Temporaire}
    	\begin{itemize}
				\item Interface pour un accès homogène aux BD:
					\begin{itemize}
						\item Le concept de Driver masque au maximum les différences des SGBD.
						\item API de bas niveau: il faut connaître SQL.
					\end{itemize}
				\item Tous les éditeurs proposent un driver JDBC.
				\item Problèmes :
					\begin{itemize}
						\item JDBC n'effectue aucun contrôle sur les instructions SQL qui sont entièrement interprétées par le SGBD.
						\item Implique l'existence de SQL dispersé dans du code Java (application polyglotte).
						\item Correspondance entre classes Java et relations est un travail de bas niveau.
					\end{itemize}
			\end{itemize}
    \end{frame}
        
  \section{Persistance fondée sur les concepts objets}
  
\end{document}

